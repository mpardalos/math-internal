\documentclass[12pt, titlepage]{article}

\usepackage[margin=1in]{geometry}
\usepackage[all]{nowidow}
\usepackage[
    citestyle=authortitle,
    autocite=footnote
]{biblatex}
    \addbibresource{bibliography.bib}
\usepackage{float}
\usepackage{amsfonts}
\usepackage{amssymb}
\usepackage{amsmath}
    \numberwithin{equation}{section}
\usepackage{amsthm}
    \newtheorem{theorem}{Theorem}
    \newtheorem*{definition}{Definition}
\usepackage[
    colorlinks,
    linkcolor=black,
    citecolor=black,
    urlcolor=blue
]{hyperref}
\usepackage{mdframed}

\setlength{\parskip}{1em}
\setlength{\parindent}{0em}

\let\oldsection\section
\renewcommand\section{\clearpage\oldsection}

\renewcommand{\geq}{\geqslant}
\renewcommand{\Re}{\text{Re}}
\renewcommand{\Im}{\text{Im}}
\newcommand{\ahalf}{\frac{1}{2}}

% commonly used integrals
\newcommand{\piint}{\int_{-\pi}^{\pi}} % -pi to pi
\newcommand{\infint}{\int_{-\infty}^{\infty}} % -infinity to infinity

\title{How can a tiny mp3 player store thousands of songs}
\author{Michalis Pardalos}
\date{}

\begin{document}
\maketitle
\tableofcontents

\section{Introduction}
\section{The Fourier Series}

The Fourier Series is defined as a series of the form
%
\begin{equation} \label{eq:fourier}
    f(t) = a_0 + \sum_{n=1}^{\infty}a_n \cos{nt} + \sum_{n=1}^{\infty}b_n \sin{nt}
\end{equation}
%
where $a_n$, $b_n$ constants. 

\subsection{Finding the coefficients} \label{sec:fourier_coefficients}

Before explaining how the Fourier coefficients are found, it will help to state the
following integral identities:\autocite[438]{courant_calculus_1}
%
\begin{align}
    \piint \sin{nx}\ dx          & = 0 \label{eq:sinint}\\
    \piint \cos{nx}\ dx          & = 0 \label{eq:cosint}\\
    \piint \sin{nx}\cos{mx}\ dx & = 0\label{eq:sincosint}\\
    \piint \cos{nx}\cos{mx}\ dx & =
    \begin{cases}
        0,   & m\not=n\\
        \pi, & m=n
    \end{cases} \label{eq:coscosint}\\
    \piint \sin{nx}\sin{mx}\ dx &= 
    \begin{cases}
        0,   & m\not=n\\
        \pi, & m=n\not=0
    \end{cases} \label{eq:sinsinint}
\end{align}
%
With these in mind, we can begin with finding some coefficients for the Fourier expansion of
a function $f(t)$:
%
\begin{equation*}
    f(t) = a_0 + \sum_{n=1}^\infty a_n \cos{nt} + \sum_{n=1}^{\infty} b_n \sin{nt}
\end{equation*}
%

First we can find $a_0$ simply by integrating with respect to $t$ over $[-\pi, \pi]$:
%
\begin{equation*}
    \piint f(t)\ dt = a_0\piint dt + \sum_{n=1}^\infty a_n \piint \cos{nt}\ dt 
        + \sum_{n=1}^{\infty} b_n \piint \sin{nt}\ dt\\
\end{equation*}
%
By equations \eqref{eq:sinint} and \eqref{eq:cosint}, this can be simplified to:
%
\begin{equation*}
    \piint f(t)\ dt &= 2\pi a_0\\
\end{equation*}
%
So, $a_0$ can be found as:
%
\begin{mdframed}
    \begin{equation}
        a_0 &= \frac{1}{2\pi}\piint f(t)\ dt
    \end{equation}
\end{mdframed}
%

Next, for $a_n$, say we want to find some $a_m$. Multiplying by $\cos{mt}$ and integrating
with respect to $t$ over $[-\pi, \pi]$ gives us:
%
\begin{equation*}
    \piint f(t)\cos{mt}\ dt =  a_0\piint \cos{mt}\ dt 
        + \sum_{n=1}^\infty a_n \piint \cos{nt}\cos{mt}\ dt 
        + \sum_{n=1}^{\infty} b_n \piint \sin{nt}\cos{mt}\ dt
\end{equation*}
%
Using equations \eqref{eq:cosint}, \eqref{eq:sincosint} and \eqref{eq:coscosint} this is reduced to:
%
\begin{equation*}
    \piint f(t)\cos{mt}\ dt = \pi a_m
\end{equation*}
%
So, any given $a_m$ can be found as:
%
\begin{mdframed}
    \begin{equation}
        a_m = \frac{1}{\pi}\piint f(t)\cos{mt}\ dt
    \end{equation}
\end{mdframed}
%

Similarly, to find some $b_m$, $m > 0$, we can multiply $f(t)$ by $\sin{mt}$ and integrate
with respect to $t$ over $[-\pi, \pi]$ to get:
%
\begin{equation*}
    \piint f(t)\sin{mt}\ dt = a_0\piint \sin{mt}\ dt 
        + \sum_{n=1}^\infty a_n \piint \cos{nt}\sin{mt}\ dt 
        + b_n \piint \sin{nt}\sin{mt}\ dt
\end{equation*}
%
Using equations \eqref{eq:sinint}, \eqref{eq:sincosint} and \eqref{eq:sinsinint}, we can see
that
%
\begin{mdframed}
    \begin{equation}
        b_m = \frac{1}{\pi}\piint f(t)\sin{mt}\ dt
    \end{equation}
\end{mdframed}

\subsection{Simplifying the Series using Euler's Formula}

While the coefficients are relatively simple to find once someone has the formulas in
\autoref{sec:fourier_coefficients}, it is often cumbersome to find 3 coefficients. We can
simplify the series to use just one coefficient using Euler's Formula.

Euler's Formula states that
%
\begin{equation} \label{eq:euler}
    e^{ix} = \cos{x} + i\sin{x}
\end{equation}
%
This can also be restated as 
%
\begin{equation} \label{eq:neg_euler}
    e^{-ix} = \cos{x} - i\sin{x}
\end{equation}
%
Adding or subtracting equations \eqref{eq:euler} and \eqref{eq:neg_euler} gives,
respectively:
%
\begin{align*}
    \cos{x} &= \frac{e^{ix} + e^{-ix}}{2} \label{eq:euler_cos}\\ 
    \sin{x} &= \frac{e^{ix} - e^{-ix}}{2i} \label{eq:euler_sin}
\end{align*}
%

Using these formulas, the Fourier series --- as stated in \autoref{eq:fourier}---
becomes:
%
\begin{equation*}
    f(t) &= a_0 + \sum_{n=1}^\infty a_n\frac{e^{int} + e^{-int}}{2} 
            + \sum_{n=1}^\infty b_n \frac{e^{int} - e^{-int}}{2i} 
\end{equation*}
%
It will make the calculations easier at this point to combine the sums. First, note that
$\frac{e^{int}+e^{-int}}{2} = 1$ for $n=0$. $a_0$ can therefore be added into the first sum,
by having it start from $n=0$. Also note that $\frac{e^{int}-e^{-int}}{2i} = 0$ for $n=0$,
so starting the second sum from $0$ does not affect the series. We can therefore combine the
series into a single sum.
%
\begin{align}
    f(t) &= \sum_{n=0}^{\infty} a_n\frac{e^{int} + e^{-int}}{2} 
            + b_n \frac{e^{int} - e^{-int}}{2i} \nonumber \\
         &= \sum_{n=0}^\infty \frac{a_n}{2}e^{int} + \frac{a_n}{2} e^{-int} 
            + \frac{b_n}{2i}e^{int} - \frac{b_n}{2i}e^{-int} \nonumber\\
         &= \sum_{n=0}^\infty \frac{a_n}{2} e^{int} + \frac{a_n}{2} e^{-int} 
            - \frac{b_n}{2} ie^{int} + \frac{b_n}{2} ie^{-int} \nonumber\\
         &= \sum_{n=0}^\infty \ahalf(a_n - ib_n)e^{int} + \ahalf(a_n + ib_n)e^{-int}
            \label{eq:complex_fourier_halfway}
\end{align}
%
Notice, that the two coefficients $a_n + ib_n$ and $a_n - ib_n$ are conjugates of each
other. We can, therefore, define a new constant $c_n$, to replace $a_n$ and $b_n$:
\autocite{fourier_lecture}
%
\begin{equation*}
    c_n = 
    \begin{cases}
        \ahalf \left(a_n - ib_n \right)        & n \geq 0\\
        \ahalf \left(a_{|n|} + ib_{|n|}\right) & n < 0
    \end{cases}
\end{equation*}
%
With this definition, continuing from \eqref{eq:complex_fourier_halfway}, we can restate
the series as:
%
\begin{equation} \label{eq:complex_fourier}
    f(t) = \sum_{n=-\infty}^{\infty}c_n e^{int}
\end{equation}
%
In this form, the series is referred to as the \emph{complex Fourier series}.

We can substitute the formulas for $a_n$ and $b_n$ into the cases for $c_n$ to obtain a
singular formula for it. Let us begin with the case of $n \geq 0$
%
\begin{align}
    c_n &= \ahalf(a_n - ib_n) \nonumber \\
        &= \ahalf \left(\frac{1}{\pi} \piint f(t)\cos{nt}\ dt 
            - i\frac{1}{\pi}\piint f(t)\sin{nt}\ dt \right) \nonumber \\
        &= \frac{1}{2\pi}\piint f(t) (\cos{nt} - i\sin{nt})\ dt \nonumber \\
        &= \frac{1}{2\pi}\piint f(t) e^{-int}\ dt \label{eq:c_n_pos_euler}
\end{align}
%

Similarly, for $n < 0$:
%
\begin{align}
    c_n &= \ahalf(a_{|n|} + ib_{|n|}) \nonumber \\
        &= \ahalf \left(\frac{1}{\pi} \piint f(t)\cos{|n|t}\ dt 
            + i\frac{1}{\pi}\piint f(t)\sin{|n|t}\ dt \right) \nonumber \\
        &= \frac{1}{2\pi}\piint f(t) (\cos{|n|t} + i\sin{|n|t})\ dt \nonumber \\
        &= \frac{1}{2\pi}\piint f(t) e^{i|n|t}\ dt \nonumber \\
        &= \frac{1}{2\pi}\piint f(t) e^{-int}\ dt & \text{Since }n<0 \label{eq:c_n_neg_euler}
\end{align}
%
Since the formulas \eqref{eq:c_n_pos_euler} and \eqref{eq:c_n_neg_euler} are the same, we
can say for all $n \in \mathbb{Z}$:
\begin{mdframed}
    \begin{equation*}
        c_n = \frac{1}{2\pi}\piint f(t) e^{-int}\ dt
    \end{equation*}
\end{mdframed}


\section{The Fourier Transform}
\section{The Discrete Fourier Transform}
\section{The Fourier Transform in mp3 compression}

\printbibliography
\end{document}
